\chapter{Viento Solar y estructuras de gran escala en el medio interplanetario}

\section*{Eyecciones de masa coronal (CME's)}
Las eyecciones de masa coronal consisten en enormes estructuras que contienen plasma y campo magnéticos que son expulsados desde el sol dentro de los limites de la heliósfera. Aunque no existe una forma directa de medir la dirección de propagación de una CME o su estructura 3-D es posible inferir ambas usando datos de los coronógrafos SECCHI a bordo de la misión STEREO.
Las CME son de gran importancia para el clima espacial porque pueden inducir fuertes tormentas magnéticas cuando interactúan con la magnetósfera terrestre. Las tormentas geomagnéticas son altamente probables a ocurrir cuando la orientación del campo magnético en el medio interplanetario (ICME) es fuerte, prolongada y de dirección mayormente hacia el sur.

Los coronógrafos nos dan una representación bidimensional de una estructura CME (3-D) que es proyectada en el plano del espacio (eje observador-sol). Las dos astronaves de STEREO orbitan al sol aproximadamente a 1 AU cerca del plano eclíptico con un ángulo de separación entre ellos creciendo a razón de 45 grados por año.
\citet{2012LRSP....9....3W}
%\citep{2012LRSP....9....3W}

\subsection*{Unidades de Brillo}
\subsection*{Importancia científica}
\subsection*{Importancia tecnológica}65