\addcontentsline{toc}{chapter}{Introducción}
%\chapter{Introducción}


%-------------------------
%-------------------------
%-------------------------


\section*{Introducción}
En la actualidad resulta imposible pensar que la humanidad pudiera tener el estilo de vida que lleva si no fuera por los avances tecnológicos que durante años han sido desarrollados; la economía, la educación y las comunicaciones son algunos de los ejemplos de la infraestructura privilegiada de nuestra era que, desde la manera mas ideal mejora la calidad de vida de todos nosotros. Pero toda esa infraestructura que se basa en satélites, DatawareHouses, Centrales de energía y dispositivos electrónicos (entre ellos computadores, teléfonos y demás gadgets) se puede ver comprometida por la amenaza constante de las variaciones del clima espacial.

Siendo nuestro Sol la mayor fuente de energía que llega a nuestro planeta, estudiar el comportamiento de el clima espacial se puede lograr analizando las propiedades del sol en función del tiempo. A pesar de que el sol emite energía en todo momento, ocasionalmente la atmósfera baja del sol eruta y envía una gran parte de su atmósfera como una corriente en chorro a grandes velocidades. Estos eventos a gran escala son conocidos como Eyecciones de Masa Coronal (CME por sus siglas en Inglés.) y pueden contener 1 000 000 000 000 kg de material y se puede desplazar por el medio interplanetario a una velocidad de mas de 1000 km/s. Si el CME es dirigido hacia la tierra, toda este material puede hacer colapsar nuestros sistemas de satélites.

La mejor forma de estudiar las CME es por medio de coronógrafos, estos instrumentos ocultan el disco solar para analizar el brillo adyacente a la corona solar, a diferencia de un telescopio que recibe toda la cantidad de brillo. Analizar el brillo que existe mas allá del disco solar es importante para el estudio de las CME porque para un observador que está en line of sight la mayor cantidad de luz proviene del sol y al ocultar este cuerpo la luz visible que se capta por el instrumento es la luz del sol que se dispersa en las partículas que componen la CME y llegan a nuestro observador descrito por una función que usa la polarización de la luz y el ángulo entre el Sol, la eyecta y el punto de observación (Dispersión de Thompson).


La observación del universo tiene sus retos y complejidades, comenzando por preguntarse: ¿qué se puede observar? ¿qué puede ser medido y bajo que condiciones? Antiguamente, las primeras observaciones que la humanidad pudo realizar eran basadas. 



\subsection*{La Luz y la Óptica Geométrica}

La observación del universo tiene sus retos y complejidades, comenzando por preguntarse: ¿qué se puede observar? ¿qué puede ser medido y bajo que condiciones? Antiguamente, las primeras observaciones que la humanidad pudo realizar eran basadas en lo que sus ojos podían ver en el firmamento: las estrellas, la forma de las constelaciones y el brillo del sol a diferentes horas del día y en diferentes épocas del año, todas estas observaciones se reducen a que forman parte de la región de sensibilidad del ojo humano hacia la luz.

Esta región se pude definir en función de la longitud de onda del espectro electromagnético (luz) que va de 400 a 700 nm y a su vez, este rango está contenido dentro de la \textit{optical window} (300 a 800 nm). Para longitudes de onda más largas que la luz visible, en la región del infrarrojo cercano, la atmósfera es bastante transparente hasta 1,3 $\mu$m. Hay algunos \textit{absorption belts} causados por agua y oxigeno molecular, pero la atmósfera se vuelve más opaca solo a longitudes de onda superiores a 1.3 $\mu$m.




%-------------------------
%-------------------------
%-------------------------

\cleardoublepage
